\documentclass[12pt]{article}
\usepackage[utf8]{inputenc}                         % UNICODE
\usepackage{amsmath}                                % Matemáticas American Mathematical Society
\usepackage{amssymb}                                % Math symbols
\usepackage{graphicx}                               % Insertar gráficos
\usepackage[ddmmyyyy]{datetime}                     %
\usepackage{bookmark}                               % Links from content index
\usepackage[spanish, es-nodecimaldot]{babel}        % Corte de las palabras en español
\usepackage{enumitem}
\usepackage[titles]{tocloft}                        % Cambiar anchura título en el índice
\usepackage{siunitx} \sisetup{ output-decimal-marker={,}, quotient-mode=fraction}   % Sistema internacional para unidades con separador decimal
\protected\def\vectori{\ensuremath{\overrightarrow{i}}}
\protected\def\vectorj{\ensuremath{\overrightarrow{j}}}
\protected\def\vectork{\ensuremath{\overrightarrow{k}}}
\usepackage{fancyhdr}                               % Cabecera de cada página
\usepackage{titling}                                % Para hacer referencia a \theauthor
%\hypersetup{hidelinks=true}
\hypersetup{
  colorlinks   = true, %Colours links instead of ugly boxes
  urlcolor     = blue, %Colour for external hyperlinks
  linkcolor    = [rgb]{0.2,0.2,0.5}, %Colour of internal links
  citecolor   = red %Colour of citations
}

\title{Fundamentos de Física III\\Solución del examen de 2015}
\date{\today}
\author{...}

\renewcommand{\thesection}{}
\renewcommand{\thesubsection}{Ejercicio \arabic{subsection}}
\renewcommand{\thesubsubsection}{\alph{subsubsection}}

\pagestyle{fancy}           % Cabecera de cada página
\fancyhf{}
\fancyhead[L]{Solución del examen de Fundamentos de Física III de 2015}
\fancyfoot[C]{\thepage}

\begin{document}
    \begin{titlepage}
    \pagenumbering{gobble}
    \maketitle
    \thispagestyle{empty}
    \hypersetup{pageanchor=true}
    \renewcommand*\contentsname{Contenidos}
    %\vspace{1cm}
    \tableofcontents
    \end{titlepage}



    \newpage
    \pagenumbering{arabic}

    % Feb 1ª Ejercicio 1 ==========================================
    \section*{Examen de la 1ª semana de febrero de 2015}
    \addcontentsline{toc}{section}{Examen de la 1ª semana de febrero de 2015}
        \subsection{} Una pequeña bacteria con una masa de aproximadamente \SI{10d-14}{\kilo\gram},
                está confinada entre dos paredes rígidas separadas $L=\SI{0.1}{\milli\metre}$
        \begin{enumerate}[label=\alph*)]
        \item Estime su velocidad mínima (cuántica) de desplazamiento. ¿Dado su orden de magnitud,
            entra el resultado dentro del ámbito clásico o cuántico? Justifique su respuesta
        \item Si, en vez de ello, su velocidad es de apróximadamente \SI{1}{\milli\metre} cada
            \SI{100}{\second}, estime el número cuántico de su estado. ¿Dado su orden de magnitud,
            entra el resultado dentro del ámbito clásico o cuántico? Justifique su respuesta.
        \end{enumerate}



        % Feb 1ª Ejercicio 2    ==========================================
        \subsection{}
        La resisitividad de la plata, con número atómico $A=\num{108}$, a una temperatura igual
        a \SI{273}{\kelvin} es \SI{1.5d-8}{\ohm\meter}, su densidad es \SI{10.5d3}{\kilo\gram\meter^-3}
        y la energía de Fermi es $R_f = \SI{5.5}{\electronvolt}$. Suponiendo que cada átomo contribuye
        en un electrón a la conducción, calcule cuánto vale el codiente entre el recorrido libre medio
        y el interespaciado atómico: $\lambda/d$.


        % Feb 1ª Ejercicio 3    ==========================================
        \subsection{}
        La vida media de los muone en reposo (tiempo propio) es \SI{2.2}{\micro\second}, mientras
        que la vida media cuando están contenidos en los rayos cósmicos se encuentra que es aproximadamente
        \SI{15}{\micro\second}. Conteste entonces a las siguientes preguntas:
        \begin{enumerate}[label=\alph*)]
        \item ¿Cuál es la velocidad de estos muones procedentes de los rayos cósmicos?
        \item ¿Cuánta distancia recorrerán antes de desintegrarse en un sistema de referencia
        en el cual su velocidad de $\num{0.6}c$?
        \item Compare la distancia del punto anterior con la distancia que el muón "ve"
        mientras está viajando.
    \end{enumerate}


    % Examen Feb 2ª semana 2015 =============================================
    % Feb 2ª Ejercicio 1    ==========================================
    \section*{Examen de la 2ª semana de febrero de 2015}
    \addcontentsline{toc}{section}{Examen de la 2ª semana de febrero de 2015}
    \setcounter{subsection}{0}
    \subsection{}
    Una bola de \SI{1.0}{\gram} puede rodar libremente dentro de un tubo de longitud $L=\SI{1.0}{\centi\meter}$.
    El tubo está tapado por ambos extremos. Si modelamos el sistema como un pozo unidimensional infinito:
    \begin{enumerate}[label=\alph*)]
        \item Determine el valor del número cuántico $n$ si damos a la bola una energía de \SI{1.0}{\milli\J}.
        \item Calcule la energía de excitación que hay que proporcionar a la bola para elevarla al siguiente nivel
        de ebergía.
        \item Comente los órdenes de magnitud obtenidos en los apartados anteriores.
    \end{enumerate}

    % Feb 2ª Ejercicio 2    ==========================================
    \subsection{}
    Suponga que una molécula diatómica tiene una energía potencial dada por:

    $$U=-\left(\frac{1}{4\pi\epsilon_0}\right)\frac{e^2}{r}+\frac{B}{r^6}$$

    con $B=\SI{1.0d-78}{\J\meter^6}$ donde $r$ es la distancia entre los centros de los dos átomos.
    Determine la separación de equilibrio esperada de los dos átomos (longitud de enlace de la molécula).


\end{document}
