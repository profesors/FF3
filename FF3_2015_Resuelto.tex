\documentclass[12pt, a4paper]{article}
\usepackage[utf8]{inputenc}                         % UNICODE
%\usepackage[T1]{fontenc}                            % Para que mi correo que tiene _ barra baja funcione al pinchar sobre él
\usepackage{amsmath}                                % Matemáticas American Mathematical Society
\usepackage{amssymb}                                % Math symbols
\usepackage{graphicx}                               % Insertar gráficos
\usepackage[ddmmyyyy]{datetime}                     %
\usepackage{bookmark}                               % Links from content index
\usepackage[spanish, es-nodecimaldot]{babel}        % Corte de las palabras en español
\usepackage{enumitem}
\usepackage[titles]{tocloft}                        % Cambiar anchura título en el índice
\usepackage{siunitx} \sisetup{ output-decimal-marker={,}, quotient-mode=fraction}   % Sistema internacional para unidades con separador decimal
\protected\def\vectori{\ensuremath{\overrightarrow{i}}}
\protected\def\vectorj{\ensuremath{\overrightarrow{j}}}
\protected\def\vectork{\ensuremath{\overrightarrow{k}}}
\usepackage{fancyhdr}                               % Cabecera de cada página
\usepackage{titling}                                % Para hacer referencia a \theauthor
%\hypersetup{hidelinks=true}


\usepackage{tocloft}
\setlength{\cftsubsecnumwidth}{5em} %   Mover el margen de los puntos (dots) en el TOC cfrXnumwidth donde X es subsec, pero también podría ser "sec", o "chap"

\title{Fundamentos de Física III \\ Soluciones de exámenes curso 2015-2016}
\date{\today}
\author{
    Samuel Gómez Fernández \\ \href{mailto:profesor_s@outlook.com}{profesor\_s@outlook.com}
    \and
    Manuel González Gálvez \\ \href{mailto:galvez29@gmail.com}{galvez29@gmail.com}
    }

\hypersetup{
  colorlinks   = true, %Colours links instead of ugly boxes
  urlcolor     = [rgb]{0.2,0.2,0.5}, %Colour for external hyperlinks
  linkcolor    = [rgb]{0.2,0.2,0.5}, %Colour of internal links
  citecolor   = red, %Colour of citations
  pdfstartview = FitBH,
  pdfpagelayout = TwoColumnLeft,
  pdfinfo = {
    Title = {\thetitle},
    Author = {\theauthor},
    Subject = {Problemas de Física Moderna},
    Keywords = {Física, UNED, Fundamentos de Física III}
    },
  pdfproducer={LaTeX},
  pdfcreator={pdfLaTeX}
}

\renewcommand{\thesection}{}
\renewcommand{\thesubsection}{Ejercicio \arabic{subsection}}
\renewcommand{\thesubsubsection}{\alph{subsubsection}}

\pagestyle{fancy}           % Cabecera de cada página
\fancyhf{}
\fancyhead[C]{Fundamentos de Física III. Soluciones de exámenes curso 2015-2016}
\fancyfoot[C]{\thepage}

\begin{document}
    \begin{titlepage}
    \pagenumbering{gobble}
    \maketitle
    \thispagestyle{empty}
    \hypersetup{pageanchor=true}
    \renewcommand*\contentsname{Contenidos}
    %\vspace{1cm}
    \tableofcontents
    \end{titlepage}



    \newpage
    \pagenumbering{arabic}

    % Feb 1ª Ejercicio 1 ==========================================
    \section*{Examen de la 1ª semana de febrero de 2015}
    \addcontentsline{toc}{section}{Examen de la 1ª semana de febrero de 2015}
        \subsection{} Una pequeña bacteria con una masa de aproximadamente \SI{10d-14}{\kilo\gram},
                está confinada entre dos paredes rígidas separadas $L=\SI{0.1}{\milli\metre}$
        \begin{enumerate}[label=\alph*)]
        \item Estime su velocidad mínima (cuántica) de desplazamiento. ¿Dado su orden de magnitud,
            entra el resultado dentro del ámbito clásico o cuántico? Justifique su respuesta
        \item Si, en vez de ello, su velocidad es de apróximadamente \SI{1}{\milli\metre} cada
            \SI{100}{\second}, estime el número cuántico de su estado. ¿Dado su orden de magnitud,
            entra el resultado dentro del ámbito clásico o cuántico? Justifique su respuesta.
        \end{enumerate}

        \textbf{Solución}
        \paragraph {a)} Se trata de un problema de ecuación de Schrödinger aplicado a un pozo unidimensional infinito
        donde su velocidad mínima será su estado de mínima energía. Por una parte su estado de mínima energía
        es
        \begin{equation}    \label{eq:E_min_caja_unidimensional_infinita}
            E_{m\acute{\i}n} = E_1 =\frac{h^2}{8mL^2}
        \end{equation}
        Por otra parte, podemos hallar su velocidad mínima despejando la
        velocidad a partir de la expresión de la energía cinética $E=\frac{1}{2}mv^2$ quedando así:
        \begin{equation}    \label{eq:vel_min_desde_energia_cinetica}
            v_{m\acute{\i}n}=\sqrt{\frac{2E_{m\acute{\i}n}}{m}}
        \end{equation}
        Sustituyendo la ecuación \ref{eq:E_min_caja_unidimensional_infinita} en la ecuación \ref{eq:vel_min_desde_energia_cinetica}
        obtenemos:
        $$
            v_{m\acute{\i}n}=\sqrt{\frac{2\frac{h^2}{8mL^2}}{m}} = \frac{h}{2mL} =
            \frac{h}{\num{2} \cdot \num{10d-14} \cdot \num{0.1d-3}} = \SI{3.313d-17}{\metre\cdot\second^{-1}}
        $$
        Es una velocidad muy baja debido a que la bacteria pertenece al mundo de la Física Clásica
        no de la Física Cuántica

        \paragraph{b)} La velocidad $v$ indicada en el enunciado es de \SI{1}{\milli\meter} cada \SI{100}{\second}, esto
        es \SI{100}{\micro\meter\cdot\second^{-1}} así pues buscamos su estádo energético que depende de su energía cinética.

        \begin{equation} \label{eq:energia_cinetica}
            E_c = E_n = \frac{1}{2}mv^2
        \end{equation}

        pero también ocurre que disponemos de otra fórmula para
        obtener la energía en el enésimo estado cuántico:

        \begin{equation}    \label{eq:estado_cuantico}
            E_n = n^2 \cdot E_1
        \end{equation}

        Igualamos las ecuaciones \ref{eq:energia_cinetica} y \ref{eq:estado_cuantico} y
        sustituimos $E_1$ por la ecuación \ref{eq:E_min_caja_unidimensional_infinita}

        \begin{equation*}
            E_n = \frac{1}{2}mv^2 = n^2 E_1 = n^2 \frac{h^2}{8mL^2}
        \end{equation*}

        De esta igualdad nos quedamos con el segundo y cuarto miembro.
        Finalmente despejamos $n$ para obtener:

        \begin{equation}
            n = \frac{2mvL}{h} = \frac{\num{2}\cdot\num{10d-14}\cdot\num{100d-6}\cdot\num{0.1d-3}}{h} \approx \num{3d12}
        \end{equation}

        Se trata de un estado cuántico muy elevado, tal y como esperabamos puesto que la bacteria pertenece a la
        Física Clásica y no a la Física Cuántica





        % Feb 1ª Ejercicio 2    ==========================================
        \subsection{}
        La resisitividad de la plata, con número atómico $A=\num{108}$, a una temperatura igual
        a \SI{273}{\kelvin} es \SI{1.5d-8}{\ohm\meter}, su densidad es \SI{10.5d3}{\kilo\gram\cdot\meter^{-3}}
        y la energía de Fermi es $E_f = \SI{5.5}{\electronvolt}$. Suponiendo que cada átomo contribuye
        en un electrón a la conducción, calcule cuánto vale el cociente entre el recorrido libre medio
        y el interespaciado atómico: $\lambda/d$.

        \textbf{Solución}
        \paragraph{} Calculemos el número de átomos por metro cuadrado teniendo en cuenta que la densidad
        debe ser dada en gramos por metro cúbico:

        \begin{equation*}
            n = \frac{N_A}{A}\cdot densidad = \frac{N_a}{108}\cdot\num{10.5d6} = \num{5.854d28} \mbox{átomos}
        \end{equation*}

        Ahora hallaremos la distancia entre átomos\footnote{Emplearemo la
        notación $d_{distancia}$ para no confundir con $d$ de densidad} $d = d_{distancia}$
        mediante la fórmula $d_{distancia}^3 = 1/n_{\acute{a}tomos}$.

        \begin{equation}    \label{eq:d_distancia1.2}
            d_{distancia} = d = \sqrt[3]{\frac{1}{n_{\acute{a}tomos}}} = \sqrt[3]{\frac{108}{N_A\cdot \mbox{densidad}}}
            = \SI{2.5753}{\angstrom}
        \end{equation}

        Finalmente debemos hallar $\lambda$ y la obtenemos despejando de la fórmula de la resistencia:
        $$\rho = \frac{m_e v_m}{n_e e^2 \lambda} \Rightarrow \lambda = \frac{m_e v_m}{n_e e^2 \rho}$$

        \begin{equation}    \label{eq:lambda_1.2}
            \lambda = \frac{m_e\sqrt{\frac{2E_f}{m_e}}}{n_e e^2 \rho} = \frac{\sqrt{2 E_f m_e}}{n_e e^2 \rho}
            = \SI{56.2124}{\nano\meter}
        \end{equation}

        La relación que nos piden es $\frac{\lambda}{d_{distancia}}$ y esta se obtiene de las ecuaciones
        \ref{eq:d_distancia1.2} y \ref{eq:lambda_1.2} quedando así:

        \begin{equation*}
            \frac{\lambda}{d_{distancia}} = \frac{\SI{56.2123}{\nano\meter}}{\SI{0.25753}{\nano\meter}}
            = \num{218.2747}
        \end{equation*}




        % Feb 1ª Ejercicio 3    ==========================================
        \subsection{}
        La vida media de los muones en reposo (tiempo propio) es \SI{2.2}{\micro\second}, mientras
        que la vida media cuando están contenidos en los rayos cósmicos se encuentra que es aproximadamente
        \SI{15}{\micro\second}. Conteste entonces a las siguientes preguntas:
        \begin{enumerate}[label=\alph*)]
        \item ¿Cuál es la velocidad de estos muones procedentes de los rayos cósmicos?
        \item ¿Cuánta distancia recorrerán antes de desintegrarse en un sistema de referencia
        en el cual su velocidad de $\num{0.6}c$?
        \item Compare la distancia del punto anterior con la distancia que el muón "ve"
        mientras está viajando.
        \end{enumerate}

        \textbf{Solución}
        \paragraph{a)} Para un observador en reposo la vida de un muón parece transcurrir a cámara lenta. Esta es
        la razón por la que el tiempo de desintegración será mayor para un observador estático.
        El tiempo se dilata en un factor $\gamma$. Sea $\tau_0 = \SI{15}{\micro\second}$ la vida
        media en reposo y $\tau = \SI{2.2}{\micro\second}$ la vida media en movimiento tenemos la
        relación $\tau_0 = \gamma \cdot \tau$, desarrollando $\gamma$ tenemos:

        $$\tau_0 = \frac{1}{\sqrt{1-\left(\frac{v}{c}\right)^2}}$$

        Nos piden la velocidad luego debemos despejamos v quedando así:


        $$v = c \cdot \sqrt{1-\left(\frac{\tau}{\tau_0}\right)^2} = c \cdot \sqrt{1-\left(\frac{11}{75}\right)^2} = \num{0.989}\cdot c$$

        que es la velocidad que pedían.

        \paragraph{b)} La distancia se puede obtener despejando $s$ de la ecuación $v=\frac{s}{t}$:

        $$s=v\cdot t = \num{0.6}\cdot c \cdot \SI{15}{\micro\second} = \SI{2700}{m}$$

        \paragraph{c)} Para el muón la longitud a altas velocidades se contrae según el \mbox{factor $\gamma$}
        $$\gamma = \frac{1}{\sqrt{1-\left(\frac{\num{0.6}\cdot c}{c}\right)^2}} = \frac{5}{4}$$

        Por otra parte, la expresión $l_0=\gamma l$ relaciona la longitud en reposo $l_0$ y la longitud para
        un observador que se mueve a velocidades relativistas $l$. Sustituimos $\gamma$ por el resultado
        obtenido y obtenemos:
        $$l=\frac{1}{\gamma}\cdot l_0 = \frac{4}{5} \cdot 2700 = \SI{2160}{\meter}$$

        que resulta ser $\num{0.8}$ veces inferior a la obtenida en el apartado anterior.






    % Examen Feb 2ª semana 2015 =============================================
    % Feb 2ª Ejercicio 1    ==========================================
    \section*{Examen de la 2ª semana de febrero de 2015}
    \addcontentsline{toc}{section}{Examen de la 2ª semana de febrero de 2015}
    \setcounter{subsection}{0}
    \subsection{}
    Una bola de \SI{1.0}{\gram} puede rodar libremente dentro de un tubo de longitud $L=\SI{1.0}{\centi\meter}$.
    El tubo está tapado por ambos extremos. Si modelamos el sistema como un pozo unidimensional infinito:
    \begin{enumerate}[label=\alph*)]
        \item Determine el valor del número cuántico $n$ si damos a la bola una energía de \SI{1.0}{\milli\J}.
        \item Calcule la energía de excitación que hay que proporcionar a la bola para elevarla al siguiente nivel
        de ebergía.
        \item Comente los órdenes de magnitud obtenidos en los apartados anteriores.
    \end{enumerate}

    % Feb 2ª Ejercicio 2    ==========================================
    \subsection{}
    Suponga que una molécula diatómica tiene una energía potencial dada por:

    $$U=-\left(\frac{1}{4\pi\epsilon_0}\right)\frac{e^2}{r}+\frac{B}{r^6}$$

    con $B=\SI{1.0d-78}{\J\meter^6}$ donde $r$ es la distancia entre los centros de los dos átomos.
    Determine la separación de equilibrio esperada de los dos átomos (longitud de enlace de la molécula).



    % Feb 2ª Ejercicio 3    ==========================================
    \subsection{}
    La función de onda de un electrón en un átomo de tipo hidrógeno en el estado fundamental
    expresada en coordenadas esféricas es:

    $$\psi(r)=\frac{1}{\sqrt{\pi a^3}}e^{-r/a}$$

    donde r es la coordenada radial, $a=a_0/Z$ y $a_0\simeq\SI{0.5}{\angstrom}$ es el radio de Bohr
    (la carga del núcleo $Z_e$ y el átomo solo incluye un electrón). Si el número total de
    nucleones es $A=173$ y el número atómico es $Z=70$:
    \begin{enumerate}[label=\alph*)]
        \item ¿Cuál es la probabilidad de que el electrón esté en el núcleo?
        \item Exprese el resultado anterior en función del número total de nucleones A.
        \item Sustituya valores y obtenga el valor numérico de esta propiedad.
    \end{enumerate}
    \textbf{Ayuda:} A la hora de calcular la integral, realice una aproximación teniendo en cuenta
    que $R\ll\alpha$ en el interior del núcleo ($R$ es el radio del núcleo).


    % Examen Sept  2016 =============================================
    % Sept 2016 Ejercicio 1    ==========================================
    \section*{Examen de septiembre de 2016}
    \addcontentsline{toc}{section}{Examen de septiembre de 2016}
    \setcounter{subsection}{0}
    \subsection{}
    Una partícula de masa $m$ y energía total nula tiene una función de onda estacionaria
    definida por

    $$\psi(x)=Axe^{-x^2/L^2}$$
    $A$ y $L$ son constantes. Determine la energía potencial $U(x)$ de la partícula.


    % Sept 2016 Ejercicio 2    ==========================================
    \subsection{}
    Los electrónes más externos de un átomo detectan un núcleo protegido o "apantallado"
    debido a lainfluencia de los electrónes intermedios. La atracción del núcleo se modela
    entonces usando un átomo de tipo hidrógeno con número atómico efectivo $Z'$. Sea entonces
    un electrón externo del átomo de sodio que ocupa el nivel atómico 3s y cuya energía
    de ionización es \SI{5.14}{\electronvolt}. Obtenga el valor de $Z'$ para ese electrón del sodio.

    % Sept 2016 Ejercicio 3    ==========================================
    \subsection{}
    El In (indio) como superconductor tiene una temperatura crítica ${T_c=\SI{3.4}{\kelvin}}$. Calcule
    lo siguiente:
    \begin{enumerate}[label=\alph*)]
        \item El gap de energía del superconductor según la teoría BCS
        \item La máxima longitud de onda del fotón que rompería el par de Cooper en el In.
    \end{enumerate}

    \textbf{Solución}
    \paragraph{a)} La solución para hallar el gap que nos piden es de aplicación directa. Simplemente hay
    que sustituir en la fórmula

    $$E_g=\frac{7}{2} k_b T_c = \SI{1.64d-22}{\J} = \SI{1.025}{\milli\electronvolt}$$

    \paragraph{b)} Para la energía obtenida en el apartado anterior debemos obtener la longitud
    de onda correspondiente a partir de $E=h\cdot f = \frac{h c}{\lambda}$

    $$\lambda = \frac{h c}{\num{1.64d-22}} = \SI{1.21d-3}{\meter}$$




    % Sept 2016 Ejercicio 4    ==========================================
    \subsection{}
    De los dos procesos de desintegración siguientes, decida cuál es posible y por qué
    (tenga en cuenta carga, número léptonico y número bariónico). Especifique a través
    de qué interacción se produce ese proceso.
    $$\Sigma^{-} \rightarrow \pi^{-} + \eta;   \quad \quad \Sigma^{-} \rightarrow \pi^{-} + p$$


\end{document}
